\chapter{Spectroscopy}
\label{chapter_spectra}


More info about previous work done with polluted WDs, but then move on to summary of: we see material in multiple ions, absorption is variable over multiple timescales.

\section{Datasets}
\label{spectra_datasets}
Info about different datasets obtained: time, duration of observations, instrumentation, reduction pipeline, special note about the dataset that coincided with a transit during the March dataset (link back to a figure in Chapter \ref{chapter_photo} showing this transit if possible).

\textbf{Work notes:} Mostly text, might have to dig into the reduction pipelines and try it for yourself. If we have HST data by then (ask Seth if it is public) talk to Seth+Julian about reduction process. Also find out if that proposal for simultaneous HST+Spitzer+VLT data ended up working out. Is that information (whether it was awarded) publicly available? If yes, which have accessible data?

\section{Fitting WD+CS}
\label{spectra_fitting}
\subsection{WD}
Look at Koester paper (find citation) about details of fitting method and this object specifically. Compare to \cite{Xu2016} and note any differences + what those differences and intrinsic uncertainties imply for conclusions we draw.

\subsection{CS}
Talk about stellar model interpolation and subtraction (if you come up with a more sophisticated way to handle removing the stellar component, replace with that. Incidentally, should look more into that. Roy also works with accreting things right? Maybe he knows other things that people try?)

Discuss trapezoidal model and what different parameters indicate while referring to a typical best-case unblended line. Find a good blended line to show where/why trapezoid fails. Then if you find a line with particularly interesting structure (double-peaked etc.), show that and talk about implication. Point out dropoff of $v_{max}$. 

Also try the Fourier transform Gaussian thing Seth mentioned for ISM work and if that ends up being a meaningful result, talk about. (Need to try it out to see if it is worth elaborating on). Look at other papers from Xu + other polluted WD people to see if they do anything special for CS absorption.

\textbf{Work notes:} Look into the simultaneous CS+WD fitting thing someone mentioned in the paper email chain and see if that actually means anything feasibly doable. The code for the majority of the CS fitting work is already done, just a matter of fine-tuning and adding on the extra things described above if they turn out to be worth doing.



\section{Ions}
\label{spectra_ions}
Look at things on the broader scale: EWs, abundances and column densities of different species averaged over all datasets or for one specific dataset. Comment on relation to bulk Earth / chondrites and other polluted WDs.

\textbf{Work notes:} Learn how to actually calculate abundances. Column density code is fine.

\section{Variability}
\label{spectra_variability}
Discuss trend in variability over datasets we've seen, see if you can quantify shifts for a particular ion by looking at EWs and $v_{min,max}$. Compare to the disk CS variability paper that just went on arxiv: \cite{Manser2016}

\textbf{Work notes:} See if any trends exist. 

\section{Disk Modeling}
\label{spectra_modeling}
Talk about various options, show examples of toy models. Then talk about Wilson's model and try working with it yourself. 

\textbf{Work notes:} Haven't done any of this, read papers and talk to Wilson to see what you can do.


% \ifthenelse{\isodd{\thepage}}{\clearemptydoublepage}{}